\documentclass[a4paper,11pt]{article}
\usepackage[spanish]{babel} % Para escribir en Espa~nol normal
\usepackage[utf8]{inputenc}
%\usepackage[latin1]{inputenc}
\usepackage{color}
\usepackage{array}
\usepackage{amsmath,amssymb}
\usepackage{float}
\usepackage{graphicx}
\usepackage{subfig}
\usepackage{enumerate}
\begin{document}

%Portada del Documento
\setlength{\unitlength}{1 cm} %Especificar unidad de trabajo
\thispagestyle{empty}
\begin{picture}(18,3)
\put(4,0){\includegraphics[width=4cm,height=5cm]{logo.jpg}}
\end{picture}
\begin{center}
\textbf{{\Huge Control de Gastos }\\[0.5cm]
{\LARGE Proyecto del Primer Parcial }}\\[1.25cm]
{\Large Lenguajes de Programación}\\[2.3cm]
{\LARGE \textbf{Observaciones, Conclusiones y experiencias }}\\[3.5cm]
\end{center}
{\Large Integrantes:}
\begin{itemize}
\item Vanessa Robles
\item Ricardo Campuzano
\item Ana Mora Ocaña
\end{itemize}
\begin{center}
 Ingeniería en Computación\\[0.3cm]
  ESPOL\\[1cm]
Guayaquil - \today
\end{center}
% fin de la portada
\newpage
\textbf{{\LARGE Observaciones y Experiencias}}

\begin{itemize}
\item En el entorno que será util la apliacación será según el lugar de encuentro del usuario en el momento de tener la factura en sus manos, porque de esta manera el usario el usuario podrá llevar un registro de sus facturas en caso de que pierda una.
\item Una observación muy importante es que nuestra aplicacion tiene uso solo para Ecuador, aunque se podri hacer cambios para  que se adapte otros paises 
\item Mediante las pruebas realizadas, los problemas que tuvieron  mayor frecuencia son:
  \begin{itemize}
\item Problemas para regresar en la aplicación.
\item Problemas para visualizar la opción de Crear Cuenta.
\item Problemas con  la información de los reportes.
   \end{itemize}
\end{itemize}


\textbf{{\LARGE Conclusiones}}

\begin{itemize}
\item  Esta experiencia ha mostrado cómo es posible diseñar y aplicar lo que hemos aprendido en anteriores curso complementado con la investigación que se realizó.
\item Tambien se demuestra que nuestra aplicación es muy útil para el usuario.
\item existen algunos cambios que se le podria hacera la aplicacion para ayudar aun mas al usuario.

\end{itemize}

\end {document}

